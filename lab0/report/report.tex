\documentclass[12pt]{article}

\usepackage[utf8]{inputenc}
\usepackage[paper=a4paper,margin=1in]{geometry}
\newgeometry{top=2cm,bottom=2cm,right=1cm,left=2cm}
\usepackage[T1]{fontenc}
\usepackage{newtxmath, newtxtext}
\usepackage{graphicx}
\graphicspath{ {images/} }

\begin{document}

    \begin{titlepage}
        \centering
        \begin{figure}[t]
        \centering
        \includegraphics[width=4cm]{utm2.png}
        \end{figure}
        { \scshape\Large Techincal University of Moldova \\ Factuly of Computers, Informatics and Microelectronics \par }
        
        \vspace{5cm}
        { \Huge\bfseries Report \par }
        { \Large\bfseries On Mobile applications development \\ }
        { \Large\bfseries Lab 0 }

        \vspace{5cm}
        \begin{flushright}
            Done by: Istratii Victor \\
            Verified by: Sergiu Ciudin
        \end{flushright}

        \vfill

        Chişinău 2017
        
    \end{titlepage}


    \newpage


    \section*{Task}
        Choose a mobile platform and explain your choice.

    \section*{The choice}
        I have chosen the \textbf{Android} platform.
        \subsection*{Technical point of view}
            The first and the most major pro to consider about Android development is the accessibility. Everyone can literally jump into the mobile development industry just by having a modern computer and internet connection. The fact is that Android is completely free, with open documentation. Unlike iOS, for developing android one does need to have a MAC computer in order to use the Apple development tools. Also, to develop publish the applications, a developer has to only pay 25\$ on Google Play market, which is a lot less, comparing to the 100\$ a year fee on AppStore\cite{androidvsios}. Android Studio is compatible with most major operating systems like Windows, OS X, and Linux. Android applications can be developed on a Windows machine, Linux, or on a Mac.
            \\\\
            For iOS development, a predefined set of standards and rules listed by Apple must be followed in order for an app to be approved and published in the App Store. In Android, a developer has more freedom and can explore different solutions for the same issue, e.g. writing custom views or modifying native views. This isn’t restricted in Android, but in iOS this kind of flexibility is off limits. 
            \\\\
            Learning Android provides a deeper understanding of how things work in mobile development. It gives you more control compared with iOS, allowing you to programmatically add/remove and edit your screens, and write screens in XML from scratch, all of which give you a better understanding of a mobile operating system’s workings. In iOS, you have less control over modifying default behaviors of views/applications, and screens are built with a graphical interface (drag and drop).

            Android also gives developers more freedom to solve problems, and allows the developer to do so using whatever custom or native approach he wants. For example, suppose you want to implement Tabs in your application. In Android, you have options for how you want to add tabs—to the Action Bar, via TabbedView, or at the top of bottom of the app—but in iOS, you have fewer options.

            Android gives an open choice to new developers. It allows you to test and distribute your application on the Google Play store or any other medium if preferred. Although Android development takes more time to master compared to iOS, its complexity can boost your confidence to master other mobile platforms and tools like iOS or Xamarin\cite{android2}.

        \subsection*{Personal point of view}
            I personally do prefer Android really because of the circumstances. I possess a Linux machine with only 4gb of RAM, which is better suited for Andrdoid development. Also, I have an android device, where I can also test the created app for compatibility with older Andrdoid versions.


    \newpage

    \section*{Conclusion}
        To be fair, I think that it will never be possible to choose between iOS or Android a better system for development. It is to mention that both systems have a lot of pros and cons, which don't allow a disbalance on the market. There will always be a vaste iOS and Android community. This concurrence will only increase the rate of improving of each system. Thus, it seems that it's up to personal thoughts and impressions, when it comes to choosing the system for development.



    \begin{thebibliography}{9}
        \bibitem{androidvsios} 
        Android vs iOS: Which Should I Learn First?
        \\\texttt{https://www.upwork.com/hiring/mobile/android-vs-ios-which-to-learn-first/}

        \bibitem{android2}
        iOS vs. Android: Which Platform Should You Build For?
        \\\texttt{https://www.androidauthority.com/developing-for-android-vs-ios-697304/}
        \end{thebibliography}

\end{document}